%Copyright (C) 2010, shiv Shankar Dayal
%This file is part of Learn Correct programming.

%This book is free document. All conditions of GNU FDL are applicable.

%YOu should have received a copy of the GNU FDL 
%along with this software.  If not, see <http://www.gnu.org/licenses/>.
%Disclaimer: Any damage or loss of data because of programs given in this
%will bear no responsibility to author under any circumstances.

\useURL[email][mailto://shivshankar.dayal@gmail.com][][shivshankar.dayal@gmail.com]
\title{Preface}
Please let me introduce myself. As you know my name is Shiv Shankar Dayal if
you have taken care of reading author's name on the first page which I expect
you have unless you were too busy and having your attention anywhere else.
I am a programmer by choice and was so by profession as well till recent past
when I decided to quit job because of personal problems. So I decided to write
this book to pass my time as I am at home full time. There are many books on C,
however, for a person like me who has no source of income is forced to rely
on open-source books. I agree that there are tons of excellent free (free as in
freedom, not free beer) software, but few free books. Again, there is a problem
that software can be created and copied and maintained easily but such is not
the case with books if you publish it. Therefore, I have decided to keep under
GNU FDL and distribute only soft copies. If some publisher publishes it even
then soft copy will remian free.

There is another reason for writing a book on C. First, I think I know it well.
Just to brag I have been programming for 11 years, however, my knowledge is
very limited and inaccurate when it comes to programming. On the contrary I
write programs very slow. They are never commented and complex and difficult
to maintain. Second, C is kind of greatest common denominator of many major
software. Almost all Unix-like systems and also Windows have been developed
in C. Programming languages like C++, Java, Perl, Python, Ruby, Lua etc have all
there roots somewhere in C. Again, I am not sure facts please cross-check
about names. I am just guessing. All the languages have C bindings. Backward
compatibility with C is a very important reason for the success of C++.
However, that is again a matter of entirely different discussion. I am just
saying what I think.

This is a book on C as you already know by its name. However, this book will
delve into topics not ventured together in many books. The most authoritative
source on this subject is ofcourse K\&R's \quotation{The C Programming
Language}. This book is not a replacement of the above book. It is a complement
and supplement for the K\&R's (henceforth I will refer it as The C Book) book.
However, C has undergone revision and latest standard being C99 which is not
part of The C Book. Therefore, until a revision is made in the contents it
will not reflect the changes in the standard. Several compilers are already
supporting this standard. \type{gcc} already supports most of standard which is
used in preparing this book.

I believe in practical programs so sometimes code examples may span several
files and pages. Please skip them if you are not interested. Sometimes toy
examples are not simple.

My thoughts jump from one topic to another and I capture them immediately so
the book may seem rather wayward. I have tried my best to organize the topics
but still the ordering may be not so good. Please use index. After all, that 
is what indices are meant for! :-)

Let us come back to the title. As you have seen it is \quotation{Learn Correct
Programming}. There are various ways to program. However, programming paradigms
associated with language does not enforce correctness in the sense of correct
way. Two programming paradigms which are generic try to do it. One of them is
\quotation{Literate Programming} and \quotation{Pair Programming}. Literate
Programming is baby of greatest computer scientist alive in my humble opinion
Donald Erwin Knuth. I am a big fan of Literate Programming, however, I can not
enforce documentation or learning \TeX{} on a normal programmer. Pair
programming is not always feasible and not particularly so in enterprise
environment where the sole aim is to make money. However, if you can
please use both of them as they will help you learn programming
immensely. Programming requires discipline. Discipline in choosing
correct language for the task, correct selection of algorithm, correct
implementation and never to mention documentation and testing.

As you may notice there is something called Part I C in this
book. There is a reason that once this part will be finished I will
put more parts probably C++ or some library uses like pthreads.

\subject{Prerequisite}
As such there are not many prerequisite. This book is well suited for
beginners as well as advanced programmers. Since I have tried to
follow C99 standard the terminology may be a bit different. I expect
the reader to be at least having high-school level knowledge of
Mathematics. If the reader does not have this prerequisite then I
strongly recommend him to study it as some of the sections may be
difficult for him to comprehend. Programmers from any language will be
able to make rapid advances while studyting the text and may even skip
some sections or chapters entirely.

\subject{Tools}
You must choose your tools carefully as it may determine the direction
your career is going in. If you want to Windows programmer choose MS
Visual C++ Express Edition if that suits you else get cygwin or MSYS
and MINGW. However, you can also choose GNU/Linux (free comes in many
variations), MacOS X (not free and supposed to run only on Mac
hardware but internally is Unix), OpenSolaris but at this moment it is
lagging as only 2009 release is there.

As compiler I use \type{gcc}, \type{gdb} as debugger, \type{emacs}
along with viper-mode, cedet, Emacs class browser as my editor and
ide. Emacs is fantastic because I can compile and debug right there in
editor. Also, viewing manuals in Emacs is pretty easy for quick
reference. But be warned Emacs has a steep learning curve and may be
daunting who has only used Windows. It will feel better if someone
entirely new to computers uses it. Also, Emacs has an in build shell,
browser, and many other things which I do not even know. Best part is
that if you want some functionality then it can be extended using
Emacs-Lisp which is another programming language of its own type.
I also use \type{valgrind} whenever I suspect a suble memory problem
or thread problem. All these tools are licensed and cost nothing.

\subject{Acknowledgements}
The time when I am writing this book is very tough on my family and
even then they have supported me. I can not show enough gratitude to
my wife, parents and siblings. They have stood by me and my decisions
to quit job which is a big change in our lives. I can not state how
much the life of my parents have effected me.

I thank my teachers Yogendra Yadav, Hriday Narayan Singh, Satyanand
Satyarthi, Gopal, Shilesh Kumar, Praveen Kumar, Prod. T K Basu, Prof. S
Sen, and Prof. S Banerjee for imparting their knowledge.

At professional level I learnt a lot from two persons. First is Anurag
Johri and second is Vibhav Saluja. They taught me several technical
things and a lot more about life.

On the programming side there are far too many people I mena whole
Free Software Community, however, I will take some names. Donald Erwin
Knuth, creator of \TeX{}, which has been used in preparation of this
book, who is also author of many books like ``The Art of Computer
Programming'' is first among them. Richard M Stallman is second but
his contributions are more to life than software, however, that does
not mean that he has done any less contribution on software side. He
contributed to \type{gcc}, \type{gdb}, \type{emacs}, \TeX\type{Info},
\type{ELisp} and may more things. His biggest contribution is free
software movement. I also want to thank Hans Hagen who helped me with
\ConTeXt{} which I used to finally process the contents of this book.
I also used \type{vim} syntax highlighting module from Aditya Mahajan
to highlight the source code given in this book. I am most grateful to
all these nice people who believe in sharing the knowledge at source
code level and not at binary level.

English is not my native language and even though I have run spell
check over buffers there are chances that grammatical and factual
errrors are there. The reason is that it has not been proof-read and
edited. All I can guarantee is that source code given runs on my
machine. Please mail me your suggestions, errors which you have found
to me at \from[email]. I will try to give
reply to each of your mail.

\startalignment[flushright]
Shiv Shankar Dayal
\stopalignment


