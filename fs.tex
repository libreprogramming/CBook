\chapter{Following Specification}
\startcolumns[n=3,distance=2em]
  \placelist
    [section]
    [alternative=c, % a,
     interaction=all,]
\stopcolumns
%In this chapter I will begin discussion with basic data types provided by C as
%a programming language. These data types are also known as primitive data
%types. Using these types we construct further slightly complex user-defined
%data types.

%\section{Basic Data Types}
%There are six basic data types. They are \type{_Bool, char, int, float, long,
%double}. There are certain modifiers which modify the ranges of these basic
%data types. On my Ubuntu 10.04 I look at \type{/usr/include/limits.h} for the
%values of these basic types. They are shown in the table given below:
%\placetable
%[here, force][tab:basic Types]{Basic Types and Their Sizes}
%{\starttable[|l|l|l|]
%\HL
%\VL Data Type \VL Minimum Value  \VL  Maximum Value\VL\FR
%\HL
%\VL \_Bool\VL0\VL1\VL\MR
%\HL
%\VL Signed Char\VL -128\VL127\VL\MR
%\HL
%\VL Unsigned Char\VL 0\VL255\VL\MR
%\HL
%\VL Signed Short Integer\VL-32768\VL32767\VL\MR
%\HL
%\VL Unsigned Short Integer\VL 0\VL65535\VL\MR
%\HL
%\VL Signed Integer\VL-2147483648\VL2147483647\VL\MR
%\HL
%\VL Unsigned Integer\VL0\VL4294967295\VL\MR
%\HL
%\VL Signed Long\VL-2147483648\VL2147483647L\VL\MR
%\HL
%\VL Signed Long\VL-2147483648\VL2147483647\VL\MR
%\HL
%\VL Unsigned Long\VL0\VL4294967295\VL\LR
%\HL
%\stoptable}
%Always remember that these values are for my machine. If your machine
%is not the same then you need to figure that. Do not worry I will
%write a program for you.

At this point of time I would like to look back and refer to ISO/IEC
9899:TC2 which is the C99 specification with me. Since my thoughts are
random in nature I will try to follow the organization of content
given in the specification. However, note that I will not include all
the contents of specification and will exclude some of very obvious
and trivial sections/subsections. The following terms, definitions and
symbols have come from specification, however, some are omitted for
the sake of conciseness. When I will repeat the specification at times I will
do a verbatim copy just for quick reference and then add an
explanation to that.

\section{Terms, Definitions, and Symbols}
\startitemize[n]
\item\type{access.}\textreference[access]~~There are two parts of any
program. Data and
instruction. Programs are stored in file on some non-volatile storage for
example, hard disk drive, CD, DVD, tape drive. When they are executed
from non-volatile storage they are transferred to some volatile
storage typically RAM (Random Access Memory) of the computer. When a
program is executed it becomes a living entity capable of doing
something and sometimes also referred as process. So when the contents
of RAM (henceforth referred as memory) is either read or written (it
does not matter whether the value is same or new) to then it is
defined as access. Here point to be noted is that the expressions
which will not be evaluated do not access objects.
\item\type{alignment.}\textreference[alignment]~Say your program
  requires x bytes of memory then it will not be always given x bytes
  but something more. Say an object requires y bytes then it will be
  always greater or equal to y bytes. This is required so that objects
  are always located on storage boundaries that are particular
  multiples of byte address. The reason for this alignment lies in
  the efficiency of the operating system as a whole. As we know that
  on 32-bit systems data bus is 32 bits similarly on 64-bit systems
  it is 64 bits. This means in one fetch cycle (read up on this on
  some microprocessor or computer architecture book) only 32-bits
  can be fetched. 32-bits means 4 bytes. Oops! I did not tell you
  about bits and bytes conversion. Not even nibble. However, since 4
  bytes can be fetched in one cycle compiler tries to optimize the
  data in group of 4 bytes. Given below are some examples.
\blank[force,1mm]\hrule\blank[force,1mm]
\startCPP
/*
 * Author: Shiv Shankar Dayal
 * Date: July 4th, 2010
 * Description: Demonstration of structure padding and memory alignment.
 */
#include <stdio.h>

typedef struct
{
  char a;
  int b;
}A;

typedef struct
{
  char a;
  int b;
  char c;
  char d;
  char e;
  int f;
}B;

typedef struct 
{
  char x;
  char y;
  int z;
}C;

typedef struct 
{
  char x;
  int z;
  char y;
}D;

int main()
{
  A a;
  B b;
  C c;
  D d;

  printf("Size of structure %c is %d\n",'A',sizeof(a));
  printf("Size of structure %c is %d\n",'B',sizeof(b));
  printf("Size of structure %c is %d\n",'C',sizeof(c));
  printf("Size of structure %c is %d\n",'D',sizeof(d));

  return 0;
}

\stopCPP
\index{align.c}
\index{Memory Alignment Program}
\hrule\blank[force,1mm]
\startalignment[middle]
Memory Alignment Program
\stopalignment
The output is shown below:
\blank[force, 1mm]
\useexternalfigure[alignment][figs/align.png]
\placefigure{Output of
  align.c}{\externalfigure[figs/align.png][rscale=1.5]}
I am always showing screenshots of program's output but now onwards I
will not show them as they just consume space. I will also explain the
output at a later stage when you will understand \type{struct}.
\item\type{argument}\textreference[argument]~~Actually you have seen
  this when I discussed the \type{printf} and \type{scanf} functions
  in Chapter 1. Sometimes they are also called actual parameters but
  as you can see in specification ISO/IEC 9899:TC2 Section 3.3 this
  term is being deprecated. A function can have zero or more actual
  arguments and if they are more than one then each of them will be
  separated by a comma. These also apply to macros that is
  preprocessor directives when invoked like a function.
\item\type{implementation-defined-behavior}\textreference[implementation-defined-behavior]~~When
  specification does not specify how a particular element of language
  should be implemented then programs use their logic to implement
  these things and sometimes it depends on hardware as well. Behavior
  of such elements is called implementation-defined-behavior.
\item\type{locale-specific-behavior}\textreference[locale-specific-behavior]~~The
  implementation has a very generic description for it. In software
  locale typically confines its meaning to language. However, if you
  know more please e-mail me about it.
\item\type{undefined-behavior}\textreference[undefined-behavior]~~Specification
  is sometimes not sure about behavior of program as it may be out of
  scope of specification. Specification gives an example of integer
  overflow because integer overflow very much depends on word-width of
  integer which again is dependent on hardware and hence it is very
  difficult to capture in a specification.
\item\type{unspecified-behavior}\textreference[unspecified-behavior]~~When
  specification specifies two or more ways to do it and imposes no
  requirement on choosing one of them then the behavior is unspecified
  as it is not mentioned in specification.
\item\type{bit}\textreference[bit]~~Now we are encountering the most
  important entity in programming that is a bit. Specification says it
  should be large enough to store one of two values. These
  values are as mentioned in Chapter 1.
\item\type{byte}\textreference[byte]~~This is what is written in
  specification. ``addressable unit of data storage large enough to
  hold any member of the basic character set of the execution
  environment''. Since the very basic character set of C is
  \type{char} which has size
  of 1 byte. However, \type{char} is defined as an integral value
  having 256 distinct numbers. 256 being equal to $2^8$ implies that a
  byte in C is made up of 8-bits.
\item\type{character}\textreference[character]~~It is specified as a
  set of elements used for organization, control and representation of
  data.
  \startitemize[R]
\item\type{single-byte character}\textreference[single-byte
  character]~~It is specified as bit representation that fits in a
  byte. This confirms our calculation of number of bits in a byte.
\item\type{multi-byte character}\textreference[multi-byte
  character]~~It is specified as sequence of one or more bytes,
  however, when more comes then it can be more than two or three or
  any integer. But for all practical purposes UTF-16 serves the usage.
\item\type{wide character}\textreference[wide character]~~It is
  specifies as sequence of bits which can fit in an object of type
  \type{wchar_t}. Using \type{wchar_t} we can represent any symbol of
  UTF-16 character set. It is a two-byte character.
\stopitemize
\item\type{forward reference}\textreference[forward reference]~~You
  can look up element 3.11 of specification for it. By forward
  reference I mean that the complete type is not known or it is an
  {\it incomplete type} where used.
\item\type{3.12 and 3.13}\textreference[3.12 and 3.13]~~Please look
  this up in specification at respective places in specification.
\item\type{object}\textreference[object]~~It is specified as region of
  data storage in the execution environment (CPU or RAM), the content
  of which can represent values. These values are always concrete in
  nature and at any moment are finite. I say so because CPU or RAM
  will store bits which will always be some finite Boolean voltage
  levels. I will also mean this as an instance of declaration of a
  data type primitive or non-primitive data type because these
  instances will have these voltage levels in the execution
  environment. When we access \at[access] these objects then we use
  the very basic nature of these objects that is type of these objects
  and interpret the values accordingly.
\item\type{parameter}\textreference[parameter]~~These are called
  formal parameters or formal argument. The usage of word formal
  argument is deprecated. These are very similar to actual arguments
  but these acquire value only after function is entered. At the time
  of declaration or definition these have no values. Same concept of
  macros which were applicable for \at[argument].
\item\type{value}\textreference[value]~~The specification says in
  section 3.17 that value is ``precise meaning of the contents of an
  object when interpreted as having a specific type''. However, I
  would like to replace the word precise with accurate as both the
  words have different meaning and here accurate is what is more
  suitable.
\startitemize[R]
\item\type{implementation-defined
  value}\textreference[implementation-defined value]~~It is a value
  which is unspecified in specification and implementation documents
  everything about it like how it is generated.
\item\type{indeterminate value}\textreference[indeterminate
  value]~~Indeterminate values are those which can not be determined
  at any particular point of time with accuracy. For example,
  unspecified values and trap representations.
\item\type{unspecified value}\textreference{unspecified value}~~Any
  object can have multiple values one at a point of time. When
  specification does not require that implementation chooses any one
  of them particularly then it is called unspecified value.
\stopitemize
\item{$\lceil x\rceil$}\textreference[ceilx]~~This has usual meaning
  that is ceiling of x.
\item{$\lfloor x\rfloor$}\textreference[floorx]~~This has usual meaning
  that is floor of x.
\stopitemize
This concludes our discussion on terms, definitions and symbols. We
will proceed with more of specification in next chapter.
