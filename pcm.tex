%Copyright (C) 2010, shiv Shankar Dayal
%This file is part of Learn Correct programming.

%This book is free document. All conditions of GNU FDL 1.3 or later are
%applicable.

%YOu should have received a copy of the GNU FDL along with this document.  If 
%not, see <http://www.gnu.org/licenses/>.
%Disclaimer: Any damage or loss of data because of programs given in this
%will bear no responsibility to author under any circumstances.
\useURL[wikipedia][http://en.wikipedia.org/][][http://en.wikipedia.org/]
\useURL[Silicon dioxide][http://www.utas.edu.au/sciencelinks/chemincon/files/s1\_grow/s1\_soils/bonding/images/network.gif][][http://www.utas.edu.au/sciencelinks/chemincon/files/s1\_grow/s1\_soils/bonding/images/network.gif]
\useURL[ShellModel][http://envirochem.us/images/periodic/shellmodel/Si.gif][][http://envirochem.us/images/periodic/shellmodel/Si.gif]
\useURL[edm][http://envirochem.us/images/periodic/ElectronDot/Si.gif][][http://envirochem.us/images/periodic/ElectronDot/Si.gif]
\chapter{Mathematics, Physics and Chemistry}
\startcolumns[n=3,distance=2em]
  \placelist
    [section]
    [alternative=c, % a,
     interaction=all,]
\stopcolumns
Let me reiterate here that I am taking a botom-up approach for
studying this book to the reader. If you already know this or do not
wnat to see the basics of how a computer works please skip to Part II
and again in II if you are not interested in specification directlty
skip to the part where programming sub-part begins and continue doing
so.

When these three things combine  only then a computer is born else
there is no computer. This is going to be a very big chapter and only
one chapter covering all there as this books is meant for programming
not science. We will srudy Chemistry first then Physics and
Mathematics as reuquired.

As you know almost all the chips and electronic cricuitry is made up
of Silicon so let us first study about it. When images will come from 
\from[wikipedia] it will be written in footnote.

\section{Silicon}
We will first study the Chemistry of Silicon whose symbol is Si in
perodic table. Most of the images are from wikipedia so their licenses
apply for images. Silicon's atomic number is 14 which means it is a
group IV element and has got a valency of four that is it can make
four covalent bonds. You may also know its distribution in shells and
subshells and that $1s^1$ $2s^2$ $2p^6$ $3s^2$ $3p^2$. As you can see second
orbital has full octet so it should have a valency of two as the 3s
subshell is also filled. But when it reacts the two electrons mover
from 3s to 3p and then there are 4 valence electrons available for
reaction. There are certain reasons for it which involve going deep in
Chemistry which I would like to avoid.

The very pure form of this element is very important in compueter or
rather semiconductor electronic industry, Silicon is the ssecond most
abundant material element by weight on out planet (mother earth). It
has got 27,200 ppm (parts per million) and relative abundance is 2
only after Oxygen which has relative abundance of 1. So it seems god
wants us just to breath and program! :-)

Tons and tons or rather millions of tons of Silicon is produced every
year across the globe just to meet the requiement of computer and
electronic industries. The very requirement of Silicon from computer
and semiconductor inductries are very strict in nature as they need
very pure form of Silicon. Only 1 out of $10^9$ parts may be
impurity. That means 1ppb (part per billion is permitted.)!!!

\subsection{Phsical and Chemical Properties}
For the curious one :-) I would have liked to give a table but a list
is more suitable in my opinion. I would have been given a table if
more elements would have been there to compare side by side.
\startitemize[n]
\item {\bf Overview}
\startitemize[r]
\item Atomic Number: 14
\item Group: 14
\item Period: 3
\item Series: Metalloids(non-metal)
\stopitemize
\item {\bf Atomic Structure}
\startitemize[r]
\item Atomic Radius: 1.46 \Angstrom
\item Atomic Volume: 12.1 $cm^3$/mol
\item Covalent Radius: 1.11 \Angstrom
\item Cross Section: (Thermal Neutron Capture) $\sigma_a$/barns:0.171
\item Crystal Structure: Cubic Face Centered
\item Electronic Configuration: $1s^2$ $2s^2$ $2p^2$ $3s^2$ $3p^2$
\item Electrons per enerfy level: 2 8 4
\item Shell Model: \useexternalfigure[sio2][figs/sio2/shellmodel.gif]
\placefigure{Shell Model Stucture of
Silicon\footnote[1]{\from[ShellModel]}}{\externalfigure[figs/sio2/shellmodel.gif][rscale=1.50]}\
\item Ionic Radius: 0.4 \Angstrom
\item Filling Orbital: $3p^2$
\item No.of Electron (with no charge): 14
\item Number of Neutrons (most common/stable nuclide): 14
\item Number of Protons: 14
\item Oxidations: 4
\item Valence Electron: $3s^2$ $3p^2$
Electtron Dot Model
\hfil\break
\blank[1mm]
\useexternalfigure[sio2][figs/sio2/edm.gif]
\placefigure{Electron Dot Model of
Si\footnote[2]{\from[edm]}}{\externalfigure[figs/sio2/edm.gif][rscale=1.50]}
\stopitemize
\startitemize[r]
\item {\bf Chemical Properties}
\item Electrochemical Equivalent: 0.26197 g/amp-hr
\item Electron Work Funciton: 4.52 eV
\item Heat of Fusion: 50.55 KJ/mol
\item First Ionization Potential: 8.151
\item Second Ionization Potential: 16.345
\item Third Ionization Potential: 33.492
\stopitemize
\stopitemize
The most commons source of Silicon exist as Silicon sioxide $[SiO_2]$
which is found as sand and quartz. As you may be knowing in $CO_2]$
there are two pi bonds and two sigma bonds that is $p\pi - p\pi$ bonds
and fills its ocet. Because of this reason $CO_2$ can form a discrete
molecule and hence is in gaseous form at room tempreature. The bonds
in $SiO_2$ are all $\sigma$ bonds.It has a 3-dimensional infinte
crystal structure like diamon. Given below is one such diagram for
$SiO_2$. You will be able to clearly see how Silicon and Oxygen atoms
form an infinte 3-dimensional crystal in space. $SiO_2$ has been found
to exist in at-least 12 different forms.. The most common ones are quartz,
trydymite and cristobalite. Each of these have different strutures at
different temperatures.
\hfil\break
\blank[1mm]
\useexternalfigure[sio2][figs/sio2/sio2.gif]
\placefigure{Crystalline Stucture of Silicon
  dioxide\footnote[3]{\from[Silicon
      dioxide]}}{\externalfigure[figs/sio2/sio2.gif][rscale=1.50]}
$\alpha$-Quartz is the is most common and is a major constituent of
granite and sandstone. Pure Silicon dioxide is colorless but
impurities may give it as color. Birthstones are colored because of
these impurities.
\subsection{Sources}
Silicon
