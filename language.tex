%Copyright (C) 2010, shiv Shankar Dayal
%This file is part of Learn Correct programming.

%This book is free document. All conditions of GNU FDL are applicable.

%YOu should have received a copy of the GNU FDL 
%along with this software.  If not, see <http://www.gnu.org/licenses/>.
%Disclaimer: Any damage or loss of data because of programs given in this
%will bear no responsibility to author under any circumstances.

\chapter{Language}
\startcolumns[n=3,distance=2em]
  \placelist
    [section]
    [alternative=c, % a,
     interaction=all,]
\stopcolumns
Finally, I can breathe relief. Explaining last chapter was not fun at
all because that involved very little code. However, this is going to
be the main chapter which will describe most of the things about the
language.

\section{Concepts}
\subsection{Scope of Identifiers}
Here we are going to talk about scope of identifiers. First question
is what an indentifier. As its name implies it identifies. Next
question is what does it identify. Well, an identifier can identify
any element of programming language; here I will restrict the notation
to identntification of an object; a function; a tag or a member of a
structure, union, or enumeration; a typedef name; a label name; a
macro name; or a macro parameter. Let us try to learn through one example.
\blank[force,1mm]\hrule\blank[force,1mm]
\startCPP
/*
 * Author: Shiv Shankar Dayal
 * Date: July 5th, 2010
 * Description: Function and block scope of identifiers.
 */

#include <stdio.h>

int main()
{
  int i=5;
  if(i==5)
    {
      int i=7;
      printf("i=%d\n",i);
    }
  printf("i=%d\n",i);

  return 0;
}

\stopCPP
\index{scope, funciton}
\index{scpoe, block}
\index{idscope.c}
\hrule
\blank[force,1mm]
\startalignment[middle]
Program for funciton block scope of identifier.
\stopalignment
If you see the output first i will be reported as 7 and second one
will be reported as 5. Here i is the identifier of \type{int}
data-type. Macro names and macro parameters can not be considered as
they are replace at the preprocessing stage itself and are not part of
symbol table but every identifer is part of symbol table. The first i
is at funciton scope and second i at block scope. We will soon see
file scope when we discuss multiple file programs.

The region for which the identifier is visible is called
scope. Suppose we would not have remove \type{int i=5;} and modified
\type{if(i==5)} to \type{if(1)} then second \type{printf} would have
generated a compile-time error. Please check this and also please use
the compilation commands with \type{--switch=c99} flag unless you are
maintaining legacy code. I hate it. The reason for second printf
reporting error is simple because it just does not know about identifier
\type{i}. There are four different scopes for identifiers and they
are: function, file, block, and function prototype. The example shown
above demonstrates the block scope of an identifier. A label name that
is label of \type{goto} is only identifier which has function scope
and this is the reson why \type{goto}s can not jump outside functions.

Structure, union, and enumeration tags have scope that begins just
after the appearance of the tag in a type specifier that declares the
tag. Each enumeration constant has scope that begins just after the
appearance of its defining enumerator in an enumerator list. Any other
identifier has scope that begins just after the completion of its
declarator.

At this point of time I think it is necessary to tell you about
certain compilation flags as programs may not involve simple concepts.
\subsubsection{Compilation Flags}
Please use following compilation flags when programming for
debugging. It is my set and you may add more.
\starttyping
-g -Wall -Werror  -std=c99 -pedantic-errors  -Wfatal-errors -Wextra
\stoptyping

I am just providing you what happens when I compile with these options
the program shown above.

\startcode
shiv@ubuntu:~/CBook/code/chap5$ gcc -v -g -Wall -Werror  -std=c99
-pedantic-errors  -Wfatal-errors -Wextra idscope.c  
Using built-in specs.
Target: i486-linux-gnu
Configured with: ../src/configure -v --with-pkgversion='Ubuntu
4.4.3-4ubuntu5'
--with-bugurl=file:///usr/share/doc/gcc-4.4/README.Bugs
--enable-languages=c,c++,fortran,objc,obj-c++ --prefix=/usr
--enable-shared --enable-multiarch --enable-linker-build-id
--with-system-zlib --libexecdir=/usr/lib --without-included-gettext
--enable-threads=posix --with-gxx-include-dir=/usr/include/c++/4.4
--program-suffix=-4.4 --enable-nls --enable-clocale=gnu
--enable-libstdcxx-debug --enable-plugin --enable-objc-gc
--enable-targets=all --disable-werror --with-arch-32=i486
--with-tune=generic --enable-checking=release --build=i486-linux-gnu
--host=i486-linux-gnu --target=i486-linux-gnu
Thread model: posix
gcc version 4.4.3 (Ubuntu 4.4.3-4ubuntu5) 
COLLECT_GCC_OPTIONS='-v' '-g' '-Wall' '-Werror' '-std=c99'
'-pedantic-errors' '-Wfatal-errors' '-Wextra' '-mtune=generic'
'-march=i486' 
 /usr/lib/gcc/i486-linux-gnu/4.4.3/cc1 -quiet -v idscope.c
 -D_FORTIFY_SOURCE=2 -quiet -dumpbase idscope.c -mtune=generic
 -march=i486 -auxbase idscope -g -Wall -Werror -pedantic-errors
 -Wfatal-errors -Wextra -std=c99 -version -fstack-protector -o
 /tmp/ccvrlj1a.s  
GNU C (Ubuntu 4.4.3-4ubuntu5) version 4.4.3 (i486-linux-gnu)
	compiled by GNU C version 4.4.3, GMP version 4.3.2, MPFR version 2.4.2-p1.
GGC heuristics: --param ggc-min-expand=100 --param ggc-min-heapsize=131072
ignoring nonexistent directory "/usr/local/include/i486-linux-gnu"
ignoring nonexistent directory "/usr/lib/gcc/i486-linux-gnu/4.4.3/../../../../i486-linux-gnu/include"
ignoring nonexistent directory "/usr/include/i486-linux-gnu"
#include "..." search starts here:
#include <...> search starts here:
 /usr/local/include
 /usr/lib/gcc/i486-linux-gnu/4.4.3/include
 /usr/lib/gcc/i486-linux-gnu/4.4.3/include-fixed
 /usr/include
End of search list.
GNU C (Ubuntu 4.4.3-4ubuntu5) version 4.4.3 (i486-linux-gnu)
	compiled by GNU C version 4.4.3, GMP version 4.3.2, MPFR version 2.4.2-p1.
GGC heuristics: --param ggc-min-expand=100 --param ggc-min-heapsize=131072
Compiler executable checksum: 5998ce5f1765e99eea5269f4c1e38d44
COLLECT_GCC_OPTIONS='-v' '-g' '-Wall' '-Werror' '-std=c99'
'-pedantic-errors' '-Wfatal-errors' '-Wextra' '-mtune=generic'
'-march=i486' 
\stopcode
\startcode
 as -V -Qy -o /tmp/ccyh2504.o /tmp/ccvrlj1a.s
GNU assembler version 2.20.1 (i486-linux-gnu) using BFD version (GNU
Binutils for Ubuntu) 2.20.1-system.20100303 
COMPILER_PATH=/usr/lib/gcc/i486-linux-gnu/4.4.3/:/usr/lib/gcc/i486-linux-gnu/4.4.3/:
/usr/lib/gcc/i486-linux-gnu/:/usr/lib/gcc/i486-linux-gnu/4.4.3/:/usr/lib/gcc/i486-linux-gnu/
:/usr/lib/gcc/i486-linux-gnu/4.4.3/:/usr/lib/gcc/i486-linux-gnu/
LIBRARY_PATH=/usr/lib/gcc/i486-linux-gnu/4.4.3/:/usr/lib/gcc/i486-linux-gnu/4.4.3/:
/usr/lib/gcc/i486-linux-gnu/4.4.3/../../../../lib/:/lib/../lib/:/usr/lib/../lib/:
/usr/lib/gcc/i486-linux-gnu/4.4.3/../../../:/lib/:/usr/lib/:/usr/lib/i486-linux-gnu/
COLLECT_GCC_OPTIONS='-v' '-g' '-Wall' '-Werror' '-std=c99'
'-pedantic-errors' '-Wfatal-errors' '-Wextra' '-mtune=generic'
'-march=i486' 
 /usr/lib/gcc/i486-linux-gnu/4.4.3/collect2 --build-id --eh-frame-hdr
 -m elf_i386 --hash-style=both -dynamic-linker /lib/ld-linux.so.2 -z
 relro /usr/lib/gcc/i486-linux-gnu/4.4.3/../../../../lib/crt1.o
 /usr/lib/gcc/i486-linux-gnu/4.4.3/../../../../lib/crti.o
 /usr/lib/gcc/i486-linux-gnu/4.4.3/crtbegin.o
 -L/usr/lib/gcc/i486-linux-gnu/4.4.3
 -L/usr/lib/gcc/i486-linux-gnu/4.4.3
 -L/usr/lib/gcc/i486-linux-gnu/4.4.3/../../../../lib -L/lib/../lib
 -L/usr/lib/../lib
 -L/usr/lib/gcc/i486-linux-gnu/4.4.3/../../.. -L/usr/lib/i486-linux-gnu
 /tmp/ccyh2504.o -lgcc --as-needed -lgcc_s --no-as-needed -lc -lgcc
 --as-needed -lgcc_s --no-as-needed
 /usr/lib/gcc/i486-linux-gnu/4.4.3/crtend.o
 /usr/lib/gcc/i486-linux-gnu/4.4.3/../../../../lib/crtn.o 
shiv@ubuntu:~/CBook/code/chap5$ 
\stopcode

The \type{-v} option was just given for verbose output else it will
show nothing as program is a valid program. Please do not include it
in normal compilations.

\subsection{Linkage of Identifiers}
We have seen identifiers and function and block scopes. Now the next
question is it is possible to make two different identifiers identify
same object or rather have them refer to same object. It is very much
possible and the concept is called linkage. There are three types of
linkages: external, internal and none.

As we have seen the static, synamic libraries and executables may or
may not contain one or more translation units. If more than one
translation units are involved in a complete program (may be a library
or executable), each declaration of each identifier with external
linkage denotes the same object or function. It is slightly difficult
to explain here as I have not yet covered some concepts like
funcitons apart from printf and scanf. Functions in C are always at
file scope. However, function from one file may call funciton residing
in another file even if both are part of two different translation
units. This is one very simple example of external linkage. Each
identiffier having internal linkage in one translation unit always
refer to same object. Example is static global variable which have
file scope. Do not worry about words static and global. Identifiers
having no linkage always refer to unique entities.

I will quote following from specification \quotation{For an identifier
declared with the storage-class specifier extern in a scope in which
a prior declaration of that identifier is visible, if the prior
declaration specifies internal or external linkage, the linkage of the
identifier at the later declaration is the same as the linkage
specified at the prior declaration. If no prior declaration is
visible, or if the prior declaration specifies no linkage, then the
identifier has external linkage.}
Say, we have a global varibale with externnal linkage and and there
exists visibility of this declaration then the linkage will be same as
prior declaration. It will be best served with an example. Please see
the following code and try to figure why it is happening even if
\type{extern int i} is at block scope in vis2.c. Remember that
visibility is controlled by scope.
\blank[force,1mm]\hrule\blank[force,1mm]
\startCPP
/*
 * Author: Shiv Shankar Dayal
 * Date: July 5th, 2010
 * Description: Linkage of identifiers.
 */

#ifndef _VIS_H_
#define _VIS_H

void f();

#endif
\stopCPP
\index{linkage+visibility}
\index{vis.h}
\hrule
\blank[force,1mm]
\startalignment[middle]
Header for scope and linkage.
\stopalignment
\blank[force,1mm]\hrule\blank[force,1mm]
\startCPP
/*
 * Author: Shiv Shankar Dayal
 * Date: July 5th, 2010
 * Description: Linkage of identifiers.
 */

#include <stdio.h>
#include <vis.h>
int i;

int main()
{
  i=5;

  f();

  printf("i=%d",i);
  return 0;
}		
\stopCPP
\index{vis1.c}
\hrule
\blank[force,1mm]
\startalignment[middle]
Iteraive version for Fibbonacci numbers.
\stopalignment
\blank[force,1mm]\hrule\blank[force,1mm]
\startCPP
/*
 * Author: Shiv Shankar Dayal
 * Date: July 5th, 2010
 * Description: Linkage of identifiers.
 */

#include <stdio.h>
#include <vis.h>

void f()
{
	extern int i;

	printf("i=%d\n",i);
	
	i=9;
}	
\stopCPP
\index{vis2.c}
\hrule
\blank[force,1mm]
\startalignment[middle]
Recursive version for Fibonacci numbers.
\stopalignment
Since I have not compiled muliple file together I give you the
command: \type{gcc -std=c99 vis1.c vis2.c -I/path-to-vis.h}. Testing
and coding for second half is left as an exercise to reader.

If the declaration of an identifier of a function has no storage-class
modifier like static, then it defaults to extern linkage. If an
object has no storage-class modifier like auto, static etc and object
has file scope then linkage is external. For example, funcitons.

An identifier declared to be anything other than an object or a
function; an identifier declared to be a function parameter; a block
scope identifier for an object declared without the storage-class
specifier extern all these have no linkage. For example a our
identifier i in program idscope.c at block level.

If, within a translation unit, the same identifier appears with both
internal and external linkage, the behavior is undefined. Writing a
program is left as an exercise to reader.

\subsection{Name Spaces of Identifiers}
It is very much possible in C to have name clashes when one
declaration of a particular identifier is visible at any point in a
translation unit, the syntactic context disambiguates 
uses that refer to different entities. The reason for these clashes is
because C does not have a feature which is there in caled
\type{namesapce}. It is one of the keywords of C++. Thus, there are
separate name spaces for various categories of identifiers, as follows: 
\startitemize[n]
\item label names (disambiguated by the syntax of the label
  declaration and use); 
\item the tags of structures, unions, and enumerations (disambiguated by following any)
  of the keywords struct, union, or enum). There are three namespaces
  possible but only one is usedl.
\item the members of structures or unions; each structure or union has a separate name
  space for its members (disambiguated by the type of the expression used to access the
  member via the . or -> operator);
\item all other identifiers, called ordinary identifiers (declared
  in ordinary declarators or as enumeration constants).
\stopitemize

\subsection{Storage Duration of Identifiers}
Storage duration is the lifetime of an object of any entity or
data-type for which it lives in memory. There are three storage
durations and they are: static, automatic and allocated. static
storage duration is achieved by static keyword. All local variables of
a funciton have automatic storage duration however you can declare
them explicitly using keyword \type{auto} but that is not required.

As long as an object's storage duration has not expired it is
guarateed that object will exist and it will have certain finite
value. It also has a constant address (that is physical memory
address). It also retains its last-stored value. If an object is
referred to outside of its lifetime then its behavior is
undefined. The value of a pointer becomess indeterminate when the
object it points to reaches the end of its lifetime, such a pointer is
also called dangling pointer (it is not a standard term). That is if
ptr is a pointer and its lifetime has expired then in most probable
case you will get a SIGSEGV or segmentation fault on Unix systems.

Any object with internal or external linkage like global variables and
static storage class variables will have static storage duration. The
lifetime of static storage duration is lifetime and initialization can
take place only once where it is declared, prior program execution.

An object's with no linkage and no static storage associated with
it has automatic storage duration. For example, variables local to a
function or block or loop counters.

For a non-variable length array type (that is no pointers), its
lifetime extends from entry into the black with which it is associated
until execution of that block ends in any way. Entering an enclosed
block or calling a function suspends, bit foes not execute the
execution of current block. Say for example,
\blank[force,1mm]\hrule\blank[force,1mm]
\startCPP
void foo()
{
/****/
}

void bar()
{
/****/
foo();
/****/
}
\stopCPP
\blank[force,1mm]\hrule\blank[force,1mm]
Then the execution of bar is suspended until foo returns after its
work. Then the execution of bar is resumed with all the objects
having same value before foo's call if not modified by foo. If a block
is entered recursively, w new instance of the object is created each
time. If the value if not initialized then initial value if
indeterminate. The initialization is done each time the blocck is
entered.

This introduces the concept of recursion so let us do it. Two famous
problems are there; Fibonacci numbers and factorial computation. I
will give both iterative version and recursive versio, Factorial is
left as an exercise for both the versions.
\blank[force,1mm]\hrule\blank[force,1mm]
\startCPP
/*
 * Author: Shiv Shankar Dayal
 * Date: July 6th, 2010
 * Description: Recursive version of Fibonacci numbers
 */

#include <stdio.h>

int main()
{
  int f1=1;  //value of first Fibonacci number
  int f2=1;  //value of second Fibonacci number
  int temp;

  int n=0;  //number of fibonacci numbers to be
            //computed other than f1 and f2

  printf("Enter the number of Fibonacci numbers to be computed\n");
  scanf("%d", &n);

  for(int i=0; i<n; i++)
    {
      temp = f2;
      f2=f1+f2;
      f1=temp;
      printf("The %dth Fibbonacci number is %d\n",i+2,f2);
    }
  return 0;
}\stopCPP
\index{vis2.c}
\hrule
\blank[force,1mm]
\startalignment[middle]
Reciursive version of Fibinacci numbers.
\stopalignment
\blank[force,1mm]\hrule\blank[force,1mm]
\startCPP
/*
 * Author: Shiv Shankar Dayal
 * Date: July 6th, 2010
 * Description: Iterative version of Fibonacci numbers
 */

#include <stdio.h>

int main()
{
  int f1=1;  //value of first Fibonacci number
  int f2=1;  //value of second Fibonacci number
  int temp;

  int n=0;  //number of fibonacci numbers to be
            //computed other than f1 and f2

  printf("Enter the number of Fibonacci numbers to be computed\n");
  scanf("%d", &n);

  for(int i=0; i<n; i++)
    {
      temp = f2;
      f2=f1+f2;
      f1=temp;
      printf("The %dth Fibbonacci number is %d\n",i+2,f2);
    }
  return 0;
}
\stopCPP
\index{itfib.c.c}
\hrule
\blank[force,1mm]
\startalignment[middle]
Iterative version of Fibinacci numbers.
\stopalignment

Fibonacci numbers are given by;
\placeformula[formula:Fibonacci]
\startformula
f_n = f_{n-1} + f_{n-2}, where f_1=f_2=0
\stopformula
Please do both iterative and recursive versions for factorials. If you
do not know or remember factorial definition then I am providing it
for quick reference:
7 20
\placeformula[formula:factorial]
\startformula
n\ = n*(n-1)!\hfill\break
\stopformula
\placeformula[formula:factzero]
\startformula
0! = 1
\stopformula

\subsection{Types}
As we know that computers being electronic device does not know the
difference between 0 and 1 so the menaing of these volatage levels or
sequence of them typically in a group of 8 (bits) or 1 byte depends
upon our level of abstraction. These groups can be stroed in a type
and the meaning will change that is abstraction will chnage if we
change the type associated with the same sequence.

Types are grouped into three types and they are: object types
i.e. types that fully describe the object, function types i.e. those
describe funcitons, and incomplete types i.e. objects that do not have
the information about their own size.

I will telll you something more about incomplete types. By size means
size of that type found by \type{sizeof} operator. It is a compile
time operator and since type is incomplete it will not work.

An object declared as type \type{_Bool} is large enough to store the
values 0 and 1. gcc defines two value 0 and 1 as well. Let us see a
trivial program.
Fibonacci numbers are given by;
\placeformula[formula:Fibonacci]
\startformula
f_n = f_{n-1} + f_{n-2}, where f_1=f_2=0
\stopformula
Please do both iterative and recursive versions for factorials. If you
do not know or remember factorial definition then I am providing it
for quick reference:
\placeformula[formula:factorial]
\startformula
n\ = n*(n-1)!\hfill\break
\stopformula
\placeformula[formula:factzero]
\startformula
0! = 1
\stopformula

\subsection{Types}
As we know that computers being electronic device does not know the
difference between 0 and 1 so the menaing of these volatage levels or
sequence of them typically in a group of 8 (bits) or 1 byte depends
upon our level of abstraction. These groups can be stroed in a type
and the meaning will change that is abstraction will chnage if we
change the type associated with the same sequence.

Types are grouped into three types and they are: object types
i.e. types that fully describe the object, function types i.e. those
describe funcitons, and incomplete types i.e. objects that do not have
the information about their own size.

I will tell you something more about incomplete types. By size means
size of that type found by \type{sizeof} operator. It is a compile
time operator and since type is incomplete it will not work.

\subsubsection{\type{_Bool}}
An object declared as type \type{_Bool} is large enough to store the
values 0 and 1. gcc defines same two values as well. Let us see a
trivial program.
\blank[force,1mm]\hrule\blank[force,1mm]
\startCPP
/*
 * Author: Shiv Shankar Dayal
 * Date: July 6th, 2010
 * Description: Demo of _Bool
 */

#include <stdio.h>

int main()
{
  _Bool a=0;

   if(a==0)
   {
     printf("I am 0\n", a);
   }

   return 0;
}
\stopCPP
\index{bool.c}
\index{\type{_Bool}}
\hrule
\blank[force,1mm]
\startalignment[middle]
Program for boolean.
\stopalignment
Howeverm you can assign any value upto 1 byte as sizeof operator gives
the size as 1 byte. In that sense its behavior is very simmilar to
char that it is made up 8 bytes and any non-ero value makes it true
i.e. equal to one. Please test it for various values and confirm the
result.

\subsubsection{\type{char}}
