%Copyright (C) 2010, shiv Shankar Dayal
%This file is part of Learn Correct programming.

%This book is free document. All conditions of GNU FDL are applicable.

%YOu should have received a copy of the GNU FDL 
%along with this software.  If not, see <http://www.gnu.org/licenses/>.
%Disclaimer: Any damage or loss of data because of programs given in this
%will bear no responsibility to author under any circumstances.

\chapter{Language}
\startcolumns[n=3,distance=2em]
  \placelist
    [section]
    [alternative=c, % a,
     interaction=all,]
\stopcolumns
Finally, I can breathe relief. Explaining last chapter was not fun at
all because that involved very little code. However, this is going to
be the main chapter which will describe most of the things about the
language.

\section{Concepts}
\subsection{Scope of Identifiers}
Here we are going to talk about scope of identifiers. First question
is what an indentifier. As its name implies it identifies. Next
question is what does it identify. Well, an identifier can identify
any element of programming language; here I will restrict the notation
to identntification of an object; a function; a tag or a member of a
structure, union, or enumeration; a typedef name; a label name; a
macro name; or a macro parameter. Let us try to learn through one example.
\blank[force,1mm]\hrule\blank[force,1mm]
\startCPP
/*
 * Author: Shiv Shankar Dayal
 * Date: July 5th, 2010
 * Description: Function and block scope of identifiers.
 */

#include <stdio.h>

int main()
{
  int i=5;
  if(i==5)
    {
      int i=7;
      printf("i=%d\n",i);
    }
  printf("i=%d\n",i);

  return 0;
}

\stopCPP
\index{scope+funciton, block+identifer}
\index{idscope.c}
\hrule
\blank[force,1mm]
\startalignment[middle]
Program for funciton block scope of identifier.
\stopalignment
If you see the output first i will be reported as 7 and second one
will be reported as 5. Here i is the identifier of \type{int}
data-type. Macro names and macro parameters can not be considered as
they are replace at the preprocessing stage itself and are not part of
symbol table but every identifer is part of symbol table. The first i
is at funciton scope and second i at block scope. We will soon see
file scope when we discuss multiple file programs.

The region for which the identifier is visible is called
scope. Suppose we would not have remove \type{int i=5;} and modified
\type{if(i==5)} to \type{if(1)} then second \type{printf} would have
generated a compile-time error. Please check this and also please use
the compilation commands with \type{--switch=c99} flag unless you are
maintaining legacy code. I hate it. The reason for second printf
reporting error is simple because it just does not know about identifier
\type{i}. There are four different scopes for identifiers and they
are: function, file, block, and function prototype. The example shown
above demonstrates the block scope of an identifier. A label name that
is label of \type{goto} is only identifier which has function scope
and this is the reson why \type{goto}s can not jump outside functions.

Structure, union, and enumeration tags have scope that begins just
after the appearance of the tag in a type specifier that declares the
tag. Each enumeration constant has scope that begins just after the
appearance of its defining enumerator in an enumerator list. Any other
identifier has scope that begins just after the completion of its
declarator.

At this point of time I think it is necessary to tell you about
certain compilation flags as programs may not involve simple concepts.
\subsubsection{Compilation Flags}
Please use following compilation flags when programming for
debugging. It is my set and you may add more.
\starttyping
-g -Wall -Werror  -std=c99 -pedantic-errors  -Wfatal-errors -Wextra
\stoptyping

I am just providing you what happens when I compile with these options
the program shown above.
\starttyping
shiv@ubuntu:~/CBook/code/chap5$ gcc -v -g -Wall -Werror  -std=c99 -pedantic-errors  -Wfatal-errors -Wextra idscope.c 
Using built-in specs.
Target: i486-linux-gnu
Configured with: ../src/configure -v --with-pkgversion='Ubuntu 4.4.3-4ubuntu5' --with-bugurl=file:///usr/share/doc/gcc-4.4/README.Bugs --enable-languages=c,c++,fortran,objc,obj-c++ --prefix=/usr --enable-shared --enable-multiarch --enable-linker-build-id --with-system-zlib --libexecdir=/usr/lib --without-included-gettext --enable-threads=posix --with-gxx-include-dir=/usr/include/c++/4.4 --program-suffix=-4.4 --enable-nls --enable-clocale=gnu --enable-libstdcxx-debug --enable-plugin --enable-objc-gc --enable-targets=all --disable-werror --with-arch-32=i486 --with-tune=generic --enable-checking=release --build=i486-linux-gnu --host=i486-linux-gnu --target=i486-linux-gnu
Thread model: posix
gcc version 4.4.3 (Ubuntu 4.4.3-4ubuntu5) 
COLLECT_GCC_OPTIONS='-v' '-g' '-Wall' '-Werror' '-std=c99' '-pedantic-errors' '-Wfatal-errors' '-Wextra' '-mtune=generic' '-march=i486'
 /usr/lib/gcc/i486-linux-gnu/4.4.3/cc1 -quiet -v idscope.c -D_FORTIFY_SOURCE=2 -quiet -dumpbase idscope.c -mtune=generic -march=i486 -auxbase idscope -g -Wall -Werror -pedantic-errors -Wfatal-errors -Wextra -std=c99 -version -fstack-protector -o /tmp/ccvrlj1a.s
GNU C (Ubuntu 4.4.3-4ubuntu5) version 4.4.3 (i486-linux-gnu)
	compiled by GNU C version 4.4.3, GMP version 4.3.2, MPFR version 2.4.2-p1.
GGC heuristics: --param ggc-min-expand=100 --param ggc-min-heapsize=131072
ignoring nonexistent directory "/usr/local/include/i486-linux-gnu"
ignoring nonexistent directory "/usr/lib/gcc/i486-linux-gnu/4.4.3/../../../../i486-linux-gnu/include"
ignoring nonexistent directory "/usr/include/i486-linux-gnu"
#include "..." search starts here:
#include <...> search starts here:
 /usr/local/include
 /usr/lib/gcc/i486-linux-gnu/4.4.3/include
 /usr/lib/gcc/i486-linux-gnu/4.4.3/include-fixed
 /usr/include
End of search list.
GNU C (Ubuntu 4.4.3-4ubuntu5) version 4.4.3 (i486-linux-gnu)
	compiled by GNU C version 4.4.3, GMP version 4.3.2, MPFR version 2.4.2-p1.
GGC heuristics: --param ggc-min-expand=100 --param ggc-min-heapsize=131072
Compiler executable checksum: 5998ce5f1765e99eea5269f4c1e38d44
COLLECT_GCC_OPTIONS='-v' '-g' '-Wall' '-Werror' '-std=c99' '-pedantic-errors' '-Wfatal-errors' '-Wextra' '-mtune=generic' '-march=i486'
 as -V -Qy -o /tmp/ccyh2504.o /tmp/ccvrlj1a.s
GNU assembler version 2.20.1 (i486-linux-gnu) using BFD version (GNU Binutils for Ubuntu) 2.20.1-system.20100303
COMPILER_PATH=/usr/lib/gcc/i486-linux-gnu/4.4.3/:/usr/lib/gcc/i486-linux-gnu/4.4.3/:/usr/lib/gcc/i486-linux-gnu/:/usr/lib/gcc/i486-linux-gnu/4.4.3/:/usr/lib/gcc/i486-linux-gnu/:/usr/lib/gcc/i486-linux-gnu/4.4.3/:/usr/lib/gcc/i486-linux-gnu/
LIBRARY_PATH=/usr/lib/gcc/i486-linux-gnu/4.4.3/:/usr/lib/gcc/i486-linux-gnu/4.4.3/:/usr/lib/gcc/i486-linux-gnu/4.4.3/../../../../lib/:/lib/../lib/:/usr/lib/../lib/:/usr/lib/gcc/i486-linux-gnu/4.4.3/../../../:/lib/:/usr/lib/:/usr/lib/i486-linux-gnu/
COLLECT_GCC_OPTIONS='-v' '-g' '-Wall' '-Werror' '-std=c99' '-pedantic-errors' '-Wfatal-errors' '-Wextra' '-mtune=generic' '-march=i486'
 /usr/lib/gcc/i486-linux-gnu/4.4.3/collect2 --build-id --eh-frame-hdr -m elf_i386 --hash-style=both -dynamic-linker /lib/ld-linux.so.2 -z relro /usr/lib/gcc/i486-linux-gnu/4.4.3/../../../../lib/crt1.o /usr/lib/gcc/i486-linux-gnu/4.4.3/../../../../lib/crti.o /usr/lib/gcc/i486-linux-gnu/4.4.3/crtbegin.o -L/usr/lib/gcc/i486-linux-gnu/4.4.3 -L/usr/lib/gcc/i486-linux-gnu/4.4.3 -L/usr/lib/gcc/i486-linux-gnu/4.4.3/../../../../lib -L/lib/../lib -L/usr/lib/../lib -L/usr/lib/gcc/i486-linux-gnu/4.4.3/../../.. -L/usr/lib/i486-linux-gnu /tmp/ccyh2504.o -lgcc --as-needed -lgcc_s --no-as-needed -lc -lgcc --as-needed -lgcc_s --no-as-needed /usr/lib/gcc/i486-linux-gnu/4.4.3/crtend.o /usr/lib/gcc/i486-linux-gnu/4.4.3/../../../../lib/crtn.o
shiv@ubuntu:~/CBook/code/chap5$ 
\stoptyping

The \type{-v} option was just given for verbose output else it will
show nothing as program is a valid program. Please do not include it
in normal compilations.

\subsection{Linkage of Identifiers}
