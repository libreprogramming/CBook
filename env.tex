%Copyright (C) 2010, shiv Shankar Dayal
%This file is part of Learn Correct programming.

%This book is free document. All conditions of GNU FDL are applicable.

%YOu should have received a copy of the GNU FDL 
%along with this software.  If not, see <http://www.gnu.org/licenses/>.
%Disclaimer: Any damage or loss of data because of programs given in this
%will bear no responsibility to author under any circumstances.

\chapter{Environment}
\startcolumns[n=3,distance=2em]
  \placelist
    [section]
    [alternative=c, % a,
     interaction=all,]
\stopcolumns
As last chapter was mapped to chapter 3 of specification similarily
this chapter is mapped to chapter 5 of specification. Please refer to
these sections simultaneously.

We store source code in two different files with two different
extendions. One regular expression for source code is *.c and the
other is *.h. These files are stored in hard disk. When compiled it
may produce *.o files optionally. Your program may also produce *.a,
*.so, a.out (this name may be different for you if you provide
\type{-o} switch to gcc. At least one *.a or *.so or a.out or its
equivalent will be produced on unix systems. On Windows systems you
may produce *.lib, *.dll or *.exe. *.lib, *.dll and *.exe map directly
to their counterparts in unix as *.a, *.so and a.out. These files are
called static library, dynamic library, and executables respectively.

As per specification we are first going to consider conceptual model
then environmental considerations.

\section{Conceptual Model}
\subsection{Translation environment}
\subsubsection{Program Structure}
A C program typically is never translated together if there are
multiple source files (i.e. several *.c files). These files are also
called {\it preprocessing files} in the specification. A source file
with all the headers included by the preprocessor directive
\type{#include} is called {\it preprocessing translation unit}. After
preprocessing, a preprocessing translation unit is called a {\it
  translation unit}. So a preprocessing translation unit can be
obntained by running \type{gcc -E filename.c} and such a translation
unit can be obtained by running \type{gcc -c filename.c}. When we run
\type{gcc -E filename.c} is sent to STDOUT i.e. monitor or screen so
if you want to see the output then you need to redirect it to a
file. You can give a command like \type{gcc -E
  filename.c>filename.extension} so that new file with name
\type{filename.extension} is saved to hard disk and can be viewed
later. When you give command \type{gcc -c filename.c} it automatically
generates corresponding \type{filename.o} unless you specify something
like \type{gcc -c filename.c -o anotherfilename.o}. This however is
not a good idea and you should refrain from it as it may lead to
filename clashes sooner or later if you try this manual option. So any
*.o file is a translation unit. It is also called object
code. Existing libraries like \type{libc.so} or \type{libm.a} is a
combination of many such translation units or *.o files. When we wrote
our first program we got an \type{a.out} file and no *.o file. Just type
\type{ldd a.out} in the directory where \type{a.out} is and you will
see what libraries it depends on. It will certainly point to
\type{libc.so}. The reason is simple. Remember our \type{printf}
function that has got its logic in this \type{libc.so} or in any other
library which \type{libc.so} depends on. You may see something like
\type{/lib/tls/i686/cmov/libc.so.6} instead of
\type{libc.so}. Therefore we conclude that at runtime either
\type{a.out} will take the definition of \type{printf} from
\type{libc.so} or at compile time itself the logic of \type{printf} is
embedded in \type{a.out}. First operation is called dynamic linking
and loading and second is called static linking. Just try \type{gcc
  -static hello.c} and \type{ldd a.out} and you will see some very
nice output. :-) Do not worry it is not going to crash your operating
system or delete any data. So by now you can infer that if we want to
do runtime linking then presence of required libraries at runtime is a
must and if we do static linking then you do not need these libraries
once they are linked with. People in general prefer dynamic linking
because a common function like \type{printf} should not have multiple
copies in memory as memory is costly and it also violates the
\quotation{Efficiency} requirement of a program. How exactly dynamic and
static linking happens will become clear only after ELF (Executable
and Linkable Format) file format is described. I will try to cover it
if time permits. ELF is a typical standard like C99 on Unix
systems. There is another format whose name you know but is slightly
older and that is \quotation{a.out}. Probably now you can guess why by
default gcc puts the executable output with filename a.out. As you
will soon see that these *.o files or translation units can be
separately translated or compiled and linked at a later stage to
produce and executable file or linkable library.

\subsubsubsection{Translation Phases}
Please refer to section 5.1.1.2 of specification. There are eight
steps specified for translation phases. I will cover how gcc
behaves. gcc is a multi-pass compiler. Several switches control its
behavior. These switches are \type{-E, -S, -c}. We have already seen
first -E switch. Let us see the -S switch. When you invoke gcc with -S
switch it generates equivalent assembly code and does not generate
object code i.e. it does not assemble or link. Given below is the
sample output of command \type{gcc -S hello.c}.
\blank[force,1mm]\hrule\blank[force,1mm]
\startASM
	.file	"hello.c"
	.section	.rodata
.LC0:
	.string	"Hello, world!"
	.text
.globl main
	.type	main, @function
main:
	pushl	%ebp
	movl	%esp, %ebp
	andl	$-16, %esp
	subl	$16, %esp
	movl	$.LC0, (%esp)
	call	puts
	movl	$0, %eax
	leave
	ret
	.size	main, .-main
	.ident	"GCC: (Ubuntu 4.4.3-4ubuntu5) 4.4.3"
	.section	.note.GNU-stack,"",@progbits
\stopASM
\index{hello.s}
\hrule
\blank[force,1mm]
\startalignment[middle]
Hellow World Program in Assembly
\stopalignment
The code shown above is in \type{as} syntax. as stands for
Assembler. Once assembly code is generated then assembler assembles it
to *.o file. Then \type{ld} the GNU linker and loader comes and links
it to a binary executable file. I think I am going a bit too wayward
as far as subject is concerned. Please see documentation for gcc, as,
and ld for complete reference. So all in all we have seen at least
four stages namely they are preprocessing with -E switch, generating
assembly code with -S, producing object code with as, and finally
linking with ld. All of this may be hidden from you or combined
together. Typically on large projects you will combine first three
steps.

\section{Diagnostics}
Section 5.1.1.3 of specification tells about implementation prodiucing
diagnostics messages in case of any problem so that the probnlems in
the source code can be sorted out. gcc sends all errors and warnings
and other diagnostics messages to \type{STDERR} device which is
typically monitor or screenm. My discussion is more becoming oriented
towards specification and gcc however this is necessary untill we hit
the stage where we can write some serious program or discuss
programming language's features.

\section{Execution Environments}
Please go through complete section 5.1.2 of specification as it is too
boring to describe everything in a book of C language. The only things
which I think must be mentioned is \type{main} function and the way it
is specified. It says the implementation must not provide a prototype
for main function. It also describes two possible ways in which main
can be written. These two ways are given below:
\blank[force,1mm]\hrule\blank[force,1mm]
\startCPP
int main(void) { /* ... */ }
int main(int argc, char *argv[]) { /* ... */ }
\stopCPP
\index{main funciton}
\hrule
\blank[force,1mm]
\startalignment[middle]
\type{main} function
\stopalignment
The block between /* and */ is supposed to contain code not only
comments. :-) The parameters shown may have different names. I
interpret \type{argc} as argument count of the arguments passed to the
program containing this main and \type{argv} as argument
value so if that is fine with you you can use them as well. There are
some more interesting things there.
\startitemize[n]
\item The value of argc will be nonnegative. This is obvious because
  we can not enter nonnegative number of arguments to the program.
\item argv[argc] shall be a \type{NULL} pointer. This will become
  clear later when we discuss array and pointers.
\item argv[0] shall be program name if the host operating system
  provides it.
\stopitemize
Rest of the points you can study yourself. I am itching to write some
code. You know I am not much of an author but a programmer who is trying
to be an author.

\section{Environmental Conditions}
Out of environmental conditions I have already described the C
alphabets or rather character sets (following the specification
language :P) in the first chapter itself. Now I am going to write some
blah, blah about Trigraph Sequences.

\subsection{Trigraph Sequences}
In this world many countries and languages are there. All countries do
not speak english and hence their character sets are also
different. But a very subtle change happens because of this and that is
that some of the keyboards have missing characters from the C
character set. Now the system has stabilized a lot but it was not so
when C was first specified. So to counter such keyboards Trigraph
sequences came into picture. Given below is a table of all trigraph
sequences. They are just replaced with equivalent characters if they
are encountered anywhere in the source code.
\placetable
[here, force][tab:Trigraph]{Trigraph Sequences}
{\starttable[|c|c|c|c|c|c|]
\HL
\VL Trigraph\VL Equivalent\VL Trigraph\VL Equivalent\VL Trigraph\VL Equivalent\VL\SR
\HL
\VL\type{??=}\VL\type{#}\VL\type{??'}\VL\type{^}\VL\type{??!}\VL\type{|}\VL\SR
\HL
\VL \type{??(}\VL\type{[}\VL\type{??)}\VL\type{]}\VL \type{??<}\VL\{\VL\MR
\HL
\VL \type{??/}\VL\backslash\VL\type{??>}\VL\}\VL\type{??-}\VL\type{~}\VL\LR
\HL
\stoptable}

One more important topic in section 5.2.2 is character display
semantics which I am going to cover next. I will be treating only item
number 2 which talks about escape sequences.
\section{Escape Sequences}
These are very important and some of them you will use very
often. These escape sequences apply to the display device or your
monitor or screen. To learn about these escape sequences it is a must
to know what is an {\it active position}. The specification specifies
that active position is the position where the next character will be
printed on the device if \type{fputc} function is used to write this
character. This usually means the position of cursor or point on your
terminal device or text editors. On a file basis they are stored as
one charcter not two.
\startitemize[n]
\item\type{\a (alert)}\textreference[alert]It should produce a visible
  or audible alert without changing the system. Typically a program
  like \type{gnome-terminal} or \type{konsole} which are both terminal
  application for GNOME and KDE are capable of producing either
  depending upon configuration.
\item\type{\b (backspace)}\textreference[backspace]It should move the
  active position to the previous position on the current line. This
  is again a requirement of terminal applications. This is required
  for backward movement of cursor or point and inserting some text
  over there. Let us write a simple program to test this.
\blank[force,1mm]\hrule\blank[force,1mm]
\startCPP
/*
 * Author: Shiv Shankar Dayal
 * Date: July 4th, 2010
 * Description: Demonstration of backspace escape sequence.
 */

#include <stdio.h>
int main()
{
  int i=0;
  printf("hello\b");
  scanf("%d",&i);

  return 0;
}
\stopCPP
\index{escape sequence+backspace}
\index{backspace.c}
\hrule
\blank[force,1mm]
\startalignment[middle]
Demonstration of backspace \backslash b escape sequemce
\stopalignment
Just watch the position of cursor on terminal. It should be over 'o' not
after it.
\item\type{\f (form feed)}\textreference[form feed]It is specified to
  move the active position to initial position of next logical
  page. Let us write another program to test it. I will just replace
  \backslash b with \backslash f.
\blank[force,1mm]\hrule\blank[force,1mm]
\startCPP
/*
 * Author: Shiv Shankar Dayal
 * Date: July 4th, 2010
 * Description: Demonstration of form feed escape sequence.
 */

#include <stdio.h>
int main()
{
  int i=0;
  printf("hello\f");
  scanf("%d",&i);

  return 0;
}
\stopCPP
\index{escape sequence+backspace}
\index{backspace.c}
\hrule
\blank[force,1mm]
\startalignment[middle]
Demonstration of form feed \backslash f escape sequence.
\stopalignment
just notice how the cursor or active position advances exactly one
line of terminal. At least that is the behavior on my machine.
\item\type{\n (new line)}\textreference[newline]~~This is simple and I
  will not provide a program for this one. The specification says the
  active position will be moved to initial position of next line. You
  have already seen this in hello.c, our first program.
\item\type{\r (carriage return)}\textreference[carriagereturn]~~This
  is similar to new line the only difference being that it moves the
  active position to initial position of current line instead of next
  line.
\item\type{\t (horizontal tab)}\textreference[tab]~~It moves active
  position by one tab horizontally. Typically it is configurable and
  of size 2, 4 and 8 characters (monospaced). This is one which you
  will encounter a lot in text editors than in terminal
  applicaitons. If the active position is at last tabulation position
  or past it then behavior is unspecified.
\item\type{\v (vertical tab)}\textreference[vtab]~~Same as horizontal
  tab just the movement is vertical.
\stopitemize

The implementation of alert, new line, carriage return, and both the
tabs is left to the reader as exercise.

\section{Signals and Interrupts}
At software level Unix systems have concept of signal which one
process can send to another and there are interrupts which are of two
types software and hardware. Please refere to seciton 5.2.3 and 5.2.4
for details as I do not want to just copy that as it is very difficult
to add an explanatiion at this moment because I have not touched
signals and interrupts at all.

\subsection{Environmental Limits}
\subsubsection{Translation Limits}
Some of these which are trivial and may need a forward reference are
just copy of specification and explaining them would require
introduction of many more terms. Specification specifies following
translation limits.
\startitemize[n]
\item 127 levels of nesting blocks. As I have told you one block is
  identified by a pari of matching braces. This means 127 nested (one
  block inside another) blocks must be possible.
\item 63 levels of nesting inclusion. I have not yet touched the
  control flow of conditional statements. The keyword \type{if} and
    \type{else} or \type{switch} come under this category.Nesting
    inclusion means one if/else inside another if/else block.
\item 12 pointer, array, and function declarators (in any
  combinations) modifying an arithemetic structure, union, or
  incomplete type in declaration. This statement isself is self
  explanatory. The difference between declaration/definition will soon
  be clear. I have used the word declarator so I will try to show its
  meaning right here. \type{Declarator. From specification if you see
    section 6.7.5 declarator means the concrete object.} However, I
  tested with gcc and it supports more that 12. But if you want to
  wrote poratble code this must not be violated.
\item 63 nesting levels of parenthesized declarators within a full
  declarator. Again it is self explanatory. Paatrethesized declarator
  is like data-type (declarator).But if you want to wrote poratble
  code this must not be violated.
\item 63 nesting levels of parenthesized expressions within a full
  expression. Plese find the given definition of expression. I will
  quote this direectly from specification so that its accurate meaning
  is not interpreted in different way. \quotation{An expression is a
    sequence of operators and operands that specifies computation of a
    value, or that designates an object or a function, or that generates
    side effects, or that performs a combination thereof.} All the
  terms which which are not clear at the moment will become clear
  soon. Implementationa may support more but if you want to
  wrote poratble code this must not be violated.
\item 63 significant initial characters in an internal identifier or a
  macro name (each universal character name or extended source
  character is considered a single character). However, certain
  implementations also allow 127 or 255 significant initial
  characters. Please refer to your compiler manual. But if you want to
  wrote poratble code this must not be violated.
\item 31 significant initial characters in an external identifier
  (each universal character name specifying a short identifier of
  0000FFFF or less is considered 6 characters, each universal
  character name specifying a short identifier of 00010000 or more is
  considered 10 characters, and each extended source character is
  considered the same number of characters as the corresponding
  universal character name, if any). Again if if you want to write
  portable code do not rely on implementation but rather stick to
  standard.
\item 4095 external identifiers in one translation unit. Once again
  same consideration should be taken for portability. Even if you
  compiler suit remains same across different platforms (hardware and
  operating system) the implementations may vary for the same vendor.
\item 511 identifiers with block scope declared in one block.
\item 4095 macro identifiers simultaneously defined in one
  preprocessing translation unit
\item 127 parameters in one function definition and similarly 127
  agrumnets in one function call. This also means printf can take only
  126 extra parameters/arguments in call.
\item 127 parameters in one macro definition or 127 arguments in one
  macro invocation. A macro is a preprocessor directive. Most famous
  is \type{#define}.
\item 4095 characters in a logical source line. Though old coding
  guidelines willl restrict you to 80 chacracters. But with modern
  graphics displays these can be more. However, Leslie Lamport the
  \LaTeX{} creator says it is difficult to read a line if it contains
  more than 70/72 characters. It is a kind of rule of typography.
\item 4095 characters in a character string literal or wide string
  literal (after concatenation)
\item 65535 bytes in an object (in a hosted environment only). This
  may be a major hinderance in developmemt. However implememtation may
  be different.
\item 15 nesting levels for \type{#include}d files. An include file
  may include another and this can be nested. However there should not
  ne more than 15 levels if you want portability.
\item 1023 case labels for a \type{switch} statement (excluding those
  for any nested \type{switch} statements). This I will describe in
  detail when we see control flows.
\item 1023 members in a single structure or union.
\item 1023 enumeration constants in a single enumeration.
\item 1023 enumeration constants in a single enumeration.
\stopitemize
You must have noticed that all the numbers are $2^n - 1$ where n is an
integer and you also know the reason why. If you are not clear then
you must wait for answer.

\subsubsection{Numerical Limits}
These limits come from specificaiton as well as compiler
implementation. Three header files contain these
details. \type{limits.h, float .h} and \type{stdint.h}. I will prepare
a table for all the limits and show you if there is a difference
between my headers and specification. I guesss there should not be any.
\placetable
[here, force][tab:Numerical Limits]{Numerical Limits}
{\starttable[|l|l|l|l| ]
\HL
\VL Macro \VL Specified Value\VL \type{limits.h} value\VL Description\VL\SR
\HL
\VL\type{CHAR_BIT}\VL 8\VL 8\VL No. of bits in \type{char} type.\VL\SR
\HL
\VL \type{SCHAR_MIN}\VL -127 or $-(2^7-1)$\note[1]\VL
(-128)\VL Minimum value of \type{signed char}\VL\SR
\HL
\VL \type{SCHAR_MIN}\VL 127 or $2^7-1$\VL 127\VL Maximum value of
  \type{signed char}\VL\SR
\HL
\VL\type{UCHAR_MAX}\VL255 or $2^8-1$\VL255\VL Minimum value of
  \type{unsigned char}\VL\SR
\HL
\VL\type{CHAR_MIN}\VL See below\note[2]\VL See below\note[2]\VL Maximum value of
  \type{unsigned char}\VL\SR
\HL
\VL\type{CHAR_MAX}\VL See below\note[2]\VL See below\note[2]\VL See below\note[2]\VL\SR
\HL
\VL\type{MB_LEN_MAX}\VL1\VL16\VL Maximum no. of bytes\note[3] in\VL\SR
\VL\VL\VL\VL multi-byte character, supporting that locale.\VL\SR
\HL
\VL \type{SHRT_MIN}\VL-32767 or -$(2^{15}-1)$\VL(-32768)\VL Minimum value
of \type{short int}\VL\SR
\HL
\VL \type{SHRT_MAX}\VL32767 or $(2^{15}-1)$\VL32767)\VL Maximum value
of \type{short int}\VL\SR
\HL
\VL \type{USHRT_MAX}\VL 65535 or $(2^{16})$\VL65535\VL Minimum value
of \type{unsigned short int}\VL\SR
\HL
\VL \type{INT_MIN}\VL-32767 or -$(2^{15}-1)$\VL\type{(-INT_MAX-1)}\VL Minimum value
of \type{int}\VL\SR
\HL
\VL \type{INT_MAX}\VL32767 or $(2^{15}-1)$\VL2147483647)\VL Maximum value
of \type{int}\VL\SR
\HL
\VL \type{UINT_MAX}\VL 65535 or $(2^{16})-1$\VL65535\VL Minimum value
of \type{unsigned int}\VL\SR
\HL
\VL \type{LONG_MIN}\VL-2147483647 or $−(2^{31} − 1)$\VL$(-LONG\_MAX - 1)$\VL Minimum value
of \type{signed long int}\VL\SR
\HL
\VL \type{LONG_MAX}\VL2147483647 or $(2^{31}-1)$\VL2147483647
\VL Maximum value of \type{signed long
  int}. Lower value is \VL\SR
\VL\VL\VL9223372036854775807 \VL for wordsize=32 and larger for
wordsize=64\VL\SR
\HL
\VL \type{ULONG_MAX}\VL 4294967295 or $(2^{32}-1)$\VL 4294967295 or \VL Minimum value
of \type{undigned long int}\VL\SR
\VL\VL\VL 18446744073709551615\VL for wordsize=32 and larger for
wordsize=64\VL\SR
\HL
\VL \type{LLONG_MIN}\VL-9223372036854775807 or $−(2^{63} − 1)$\VL$(-LLONG\_MAX - 1)$\VL Minimum value
of \type{signed long int}\VL\SR
\HL
\VL \type{LLONG_MAX}\VL 9223372036854775807 or $(2^{63}-1)$\VL9223372036854775807
\VL Minimum value of \type{signed long long int}.\VL\SR
\HL
\VL \type{LLONG_MAX}\VL 18446744073709551615\VL 18446744073709551615\VL Maximum value
of \type{undigned long long int}\VL\SR
\HL
\stoptable}
\footnote[1]{I do agree with -127 being equal to minimum value of a \type
{signed char}If we see 2's complement form then it should be -128. I
know that -128 can not be represented without MSB as most significant
9th bit. Probably gcc has got it wrong or my interpretation is incomplete.}
\footnote[2]{Char is internally treated as 1-byte integer when used in
  an expression. The value of \type{CHAR_MIN} shall be the same as that of
  \type{SCHAR_MIN} and the value of \type{CHAR_MAX} shall be the same as that of
  \type{SCHAR_MAX}. Otherwise, the value of \type{CHAR_MIN} shall be 0 and the
  value of \type{CHAR_MAX} shall be the same as that of \type{UCHAR_MAX}.}
\footnote[3]{Specification specifies no. of bytes which is one can
  accomodate only uTF-8 charset. GCC used 16 but I doubt if it is
  16 bytes. In my humble opinion it is 16-bits that is 2 bytes capable
  of representing UTF-16.}

\subsubsection{Floating-Point Numbers}
Now we have come to some topic which is slightly
involving. Representing floating-pont numbers is and always was a
difficult task because all floating-numbers do not terminate in
fixed-width binary numbers and hence when you convert them back you do
not get the exact floating-point number. Therefore a model was given
which is given below for number x:

\setupformulas[left={[},right={]},numberstyle=bold]
\placeformula
\startformula
x = sb^e\sum_{k=1}^p f_k b^{-k} , e_{min}\le e\ge_{max}
\stopformula
Given below are description of all the terms:\hfill\break
\indent $s$ sign ($\pm$)\hfill\break
\indent $b$ base or radix of an exponent representation (an integer $>$
1)\hfill\break
\indent $e$ expoenent (an integer between $e_{min}$ and $e_{max}$)\hfill\break
\indent $p$ precision (the bumber base-$b$ digits in the significant\hfill\break
\indent $f_k$ nonnegative integers less than $b$ the significant
digits.\hfill\break
In some cases subnormal floating-oint numbers ($x\ne0, e=e_{min}),
f_1=0$) and unnormalized floating-point numbers ($x=0, e\gt e_{min}),
f_1=0$ and values that are not floating-pont numbers. such as
integers, and NANs may be contained by floating-point number. A NAN is
an entity which signifies Not-a-Number. A quiet NaN propagates through
almost every arithmetic operation without raising a floating point
exception (we will see these later); signaling  NaN generally raises a
floating-point exception when occurring in as an airthmetic operand
which is quiet natural and obvious (we will see these examples in
source code). You can refer to section 5.2.4.2.2 for more information.
IEC 60559:1989 gives details about quiet and signaling NaNs. For quick
reference I am giving some details. A bit-wise example of a IEEE
floating-point standard single precision (32-bit) NaN: s111 1111 1axx
xxxx xxxx xxxx xxxx xxxx where s is the sign, x is the payload, and a
determines the type of NaN. If a = 1, it is a quiet NaN if a is zero
and the payload is nonzero, then it is a signalling NaN. Please refer
to IEEE specification bitwise implementation of floating-point numbers
as well. I have not created bibliography so I am giving names of
specificaition. Once I create it I may replace these with URLs.

The specification specifies that \type{FLT_ROUNDS} should be constant
expression suitable for \type{#if} preprocessing directive which is
defines as 1 in \type{float.h} on my machine. It also says all
floating values shall be constant expressions, which you can verify
for your system either by seieng the header file or by looking at
compiler manual. It also says all except \type{DECIMAL_DIG, FLT_EVAL,
  FLT_RADIX} and \type{FLT_ROUNDS) will have seaprate names for all
  three floating-point type. The model representation for \type
  FLT_EVAL_METHOD} and \type{FLT_ROUNDS} may not be procided for rest
it is mandatory. I talked about three types of floating-point
numbers. They are finite numbers, infinity and NaNs.

tThe specification of C languages says that The rounding mode for
floating-point addition is characterized by the implementationdefined
value of \type{FLT_ROUNDS}. It has following values:\hfill\break
\indent-1 indeterminable\hfill\break
\indent0 toward zero\hfill\break
\indent1 to nearest\hfill\break
\indent2 toward positive \infty\hfill\break
\indent3 toward negative \infty\hfill\break
All other values for \type{FLT_ROUNDS} characterize implementation-defined rounding
behavior. 

As I have already mentioned for gcc it is 1 which seems to show that
we will get to nearest values.

When an arithmetic operation is performed with floating-point numbers
or conversion is done range and precision required may be greater than
reqiured by the type. The usage of evaluation format is determined by
\type{FLT_EVAL_METHOD}. It has following values:\hfill\break
\indent -1 indeterminable;\hfill\break
\indent 0 evaluate all operations and constants just to the range and
precision of the type;\hfill\break
\indent 1 evaluate operations and constants of type float and double to the
range and precision of the double type, evaluate long double
operations and constants to the range and precision of the long double
type;\hfill\break
\indent 2 evaluate all operations and constants to the range and precision of the
long double type.\hfill\break
All other negative values for \type{FLT_EVAL_METHOD} characterize
implementation-defined behavior, which I am not going to cover. I have
not looked at source code of gcc from header it seems to be either 0
or 1. This means that it will first try in the same data type else it
will promote the data type for calculation.

I am bored because I am writing no code so at the moment I will stop
environmental discussions here and let us proceed with the actual
content that is C programming language. I promise that I will revisit
this page and update it as not much is remianing in this chapter.
