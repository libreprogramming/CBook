\chapter{Environment}
\startcolumns[n=3,distance=2em]
  \placelist
    [section]
    [alternative=c, % a,
     interaction=all,]
\stopcolumns
As last chapter was mapped to chapter 3 of specification similarily
this chapter is mapped to chapter 5 of specification. Please refer to
these sections simultaneously.

We store source code in two different files with two different
extendions. One regular expression for source code is *.c and the
other is *.h. These files are stored in hard disk. When compiled it
may produce *.o files optionally. Your program may also produce *.a,
*.so, a.out (this name may be different for you if you provide
\type{-o} switch to gcc. At least one *.a or *.so or a.out or its
equivalent will be produced on unix systems. On Windows systems you
may produce *.lib, *.dll or *.exe. *.lib, *.dll and *.exe map directly
to their counterparts in unix as *.a, *.so and a.out. These files are
called static library, dynamic library, and executables respectively.

As per specification we are first going to consider conceptual model
then environmental considerations.

\section{Conceptual Model}
\subsection{Translation environment}
\subsubsection{Program Structure}
A C program typically is never translated together if there are
multiple source files (i.e. several *.c files). These files are also
called {\it preprocessing files} in the specification. A source file
with all the headers included by the preprocessor directive
\type{#include} is called {\it preprocessing translation unit}. After
preprocessing, a preprocessing translation unit is called a {\it
  translation unit}. So a preprocessing translation unit can be
obntained by running \type{gcc -E filename.c} and such a translation
unit can be obtained by running \type{gcc -c filename.c}. When we run
\type{gcc -E filename.c} is sent to STDOUT i.e. monitor or screen so
if you want to see the output then you need to redirect it to a
file. You can give a command like \type{gcc -E
  filename.c>filename.extension} so that new file with name
\type{filename.extension} is saved to hard disk and can be viewed
later. When you give command \type{gcc -c filename.c} it automatically
generates corresponding \type{filename.o} unless you specify something
like \type{gcc -c filename.c -o anotherfilename.o}. This however is
not a good idea and you should refrain from it as it may lead to
filename clashes sooner or later if you try this manual option. So any
*.o file is a translation unit. It is also called object
code. Existing libraries like \type{libc.so} or \type{libm.a} is a
combination of many such translation units or *.o files. When we wrote
our first program we got an \type{a.out} file and no *.o file. Just type
\type{ldd a.out} in the directory where \type{a.out} is and you will
see what libraries it depends on. It will certainly point to
\type{libc.so}. The reason is simple. Remember our \type{printf}
function that has got its logic in this \type{libc.so} or in any other
library which \type{libc.so} depends on. You may see something like
\type{/lib/tls/i686/cmov/libc.so.6} instead of
\type{libc.so}. Therefore we conclude that at runtime either
\type{a.out} will take the definition of \type{printf} from
\type{libc.so} or at compile time itself the logic of \type{printf} is
embedded in \type{a.out}. First operation is called dynamic linking
and loading and second is called static linking. Just try \type{gcc
  -static hello.c} and \type{ldd a.out} and you will see some very
nice output. :-) Do not worry it is not going to crash your operating
system or delete any data. So by now you can infer that if we want to
do runtime linking then presence of required libraries at runtime is a
must and if we do static linking then you do not need these libraries
once they are linked with. People in general prefer dynamic linking
because a common function like \type{printf} should not have multiple
copies in memory as memory is costly and it also violates the
\quotation{Efficiency} requirement of a program. How exactly dynamic and
static linking happens will become clear only after ELF (Executable
and Linkable Format) file format is described. I will try to cover it
if time permits. ELF is a typical standard like C99 on Unix
systems. There is another format whose name you know but is slightly
older and that is \quotation{a.out}. Probably now you can guess why by
default gcc puts the executable output with filename a.out. As you
will soon see that these *.o files or translation units can be
separately translated or compiled and linked at a later stage to
produce and executable file or linkable library.

\subsubsubsection{Translation Phases}
Please refer to section 5.1.1.2 of specification. There are eight
steps specified for translation phases. I will cover how gcc
behaves. gcc is a multi-pass compiler. Several switches control its
behavior. These switches are \type{-E, -S, -c}. We have already seen
first -E switch. Let us see the -S switch. When you invoke gcc with -S
switch it generates equivalent assembly code and does not generate
object code i.e. it does not assemble or link. Given below is the
sample output of command \type{gcc -S hello.c}.
\blank[force,1mm]\hrule\blank[force,1mm]
\startASM
	.file	"hello.c"
	.section	.rodata
.LC0:
	.string	"Hello, world!"
	.text
.globl main
	.type	main, @function
main:
	pushl	%ebp
	movl	%esp, %ebp
	andl	$-16, %esp
	subl	$16, %esp
	movl	$.LC0, (%esp)
	call	puts
	movl	$0, %eax
	leave
	ret
	.size	main, .-main
	.ident	"GCC: (Ubuntu 4.4.3-4ubuntu5) 4.4.3"
	.section	.note.GNU-stack,"",@progbits
\stopASM
\index{hello.s}
\hrule
\blank[force,1mm]
\startalignment[middle]
Hellow World Program in Assembly
\stopalignment
The code shown above is in \type{as} syntax. as stands for
Assembler. Once assembly code is generated then assembler assembles it
to *.o file. Then \type{ld} the GNU linker and loader comes and links
it to a binary executable file. I think I am going a bit too wayward
as far as subject is concerned. Please see documentation for gcc, as,
and ld for complete reference. So all in all we have seen at least
four stages namely they are preprocessing with -E switch, generating
assembly code with -S, producing object code with as, and finally
linking with ld. All of this may be hidden from you or combined
together. Typically on large projects you will combine first three
steps.

\section{Diagnostics}
Section 5.1.1.3 of specification tells about implementation prodiucing
diagnostics messages in case of any problem so that the probnlems in
the source code can be sorted out. gcc sends all errors and warnings
and other diagnostics messages to \type{STDERR} device which is
typically monitor or screenm. My discussion is more becoming oriented
towards specification and gcc however this is necessary untill we hit
the stage where we can write some serious program or discuss
programming language's features.

\section{Execution Environments}
Please go through complete section 5.1.2 of specification as it is too
boring to describe everything in a book of C language. The only things
which I think must be mentioned is \type{main} function and the way it
is specified. It says the implementation must not provide a prototype
for main function. It also describes two possible ways in which main
can be written. These two ways are given below:
\blank[force,1mm]\hrule\blank[force,1mm]
\startCPP
int main(void) { /* ... */ }
int main(int argc, char *argv[]) { /* ... */ }
\stopCPP
\index{main funciton}
\hrule
\blank[force,1mm]
\startalignment[middle]
\type{main} function
\stopalignment
The block between /* and */ is supposed to contain code not only
comments. :-) The parameters shown may have different names. I
interpret \type{argc} as argument count of the arguments passed to the
program containing this main and \type{argv} as argument
value so if that is fine with you you can use them as well. There are
some more interesting things there.
\startitemize[n]
\item The value of argc will be nonnegative. This is obvious because
  we can not enter nonnegative number of arguments to the program.
\item argv[argc] shall be a \type{NULL} pointer. This will become
  clear later when we discuss array and pointers.
\item argv[0] shall be program name if the host operating system
  provides it.
\stopitemize
Rest of the points you can study yourself. I am itching to write some
code. You know I am not much of an author but a programmer who is trying
to be an author.

\section{Environmental Conditions}
Out of environmental conditions I have already described the C
alphabets or rather character sets (following the specification
language :P) in the first chapter itself. Now I am going to write some
blah, blah about Trigraph Sequences.

\subsection{Trigraph Sequences}
In this world many countries and languages are there. All countries do
not speak english and hence their character sets are also
different. But a very subtle change happens because of this and that is
that some of the keyboards have missing characters from the C
character set. Now the system has stabilized a lot but it was not so
when C was first specified. So to counter such keyboards Trigraph
sequences came into picture. Given below is a table of all trigraph
sequences. They are just replaced with equivalen characters if they
are encountered anywhere in the source code.
\placetable
[here, force][tab:Trigraph]{Trigraph Sequences}
{\starttable[|c|c|c|c|c|c|]
\HL
\VL Trigraph\VL Equivalent\VL Trigraph\VL Equivalent\VL Trigraph\VL Equivalent\VL\SR
\HL
\VL\type{??=}\VL\type{#}\VL\type{??'}\VL\type{^}\VL\type{??!}\VL\type{|}\VL\FR
\HL
\VL \type{??(}\VL\type{[}\VL\type{??)}\VL\type{]}\VL \type{??<}\VL\{\VL\MR
\HL
\VL \type{??/}\VL\backslash\VL\type{??>}\VL\}\VL\type{??-}\VL\type{~}\VL\LR
\HL
\stoptable}

One more important topic in section 5.2.2 is character display
semantics which I am going to cover next. I will be treating only item
number 2 which talks about escape sequences.
\section{Escape Sequences}
These are very important and some of them you will use very
often. These escape sequences apply to the display device or your
monitor or screen. To learn about these escape sequences it is a must
to know what is an {\it active position}. The specification specifies
that active position is the position where the next character will be
printed on the device if \type{fputc} function is used to write this
character. This usually means the position of cursor or point on your
terminal device.
\startitemize[n]
\item\type{\a (alert)}\textreference[alert]It should produce a visible
  or audible alert without changing the system. Typically a program
  like \type{gnome-terminal} or \type{konsole} which are both terminal
  application for GNOME and KDE are capable of producing either
  depending upon configuration.
\item\type{\b (backspace)}\textreference[backspace]It should move the
  active position to the previous position on the current line. This
  is again a requirement of terminal applications. This is required
  for backward movement of cursor or point and inserting some text
  over there. Let us write a simple program to test this.
\blank[force,1mm]\hrule\blank[force,1mm]
\startCPP
/*
 * Author: Shiv Shankar Dayal
 * Date: July 4th, 2010
 * Description: Demonstration of backspace escape sequence.
 */

#include <stdio.h>
int main()
{
  int i=0;
  printf("hello\b");
  scanf("%d",&i);

  return 0;
}
\stopCPP
\index{escape sequence+backspace}
\hrule
\blank[force,1mm]
\startalignment[middle]
Demonstration of backspace \backslash b escape sequemce
\stopalignment
Just watch the position of cursor on terminal. It should be over o not
after it.
\stopitemize

