\chapter{Environment}
\startcolumns[n=3,distance=2em]
  \placelist
    [section]
    [alternative=c, % a,
     interaction=all,]
\stopcolumns
As last chapter was mapped to chapter 3 of specification similarily
this chapter is mapped to chapter 5 of specification. Please refer to
these sections simultaneously.

We store source code in two different files with two different
extendions. One regular expression for source code is *.c and the
other is *.h. These files are stored in hard disk. When compiled it
may produce *.o files optionally. Your program may also produce *.a,
*.so, a.out (this name may be different for you if you provide
\type{-o} switch to gcc. At least one *.a or *.so or a.out or its
equivalent will be produced on unix systems. On Windows systems you
may produce *.lib, *.dll or *.exe. *.lib, *.dll and *.exe map directly
to their counterparts in unix as *.a, *.so and a.out. These files are
called static library, dynamic library, and executables respectively.

As per specification we are first going to consider conceptual model
then environmental considerations.

\section{Conceptual Model}
\subsection{Translation environment}
\subsubsection{Program Structure}
A C program typically is never translated together if there are
multiple source files (i.e. several *.c files). These files are also
called {\it preprocessing files} in the specification. A source file
with all the headers included by the preprocessor directive
\type{#include} is called {\it preprocessing translation unit}. After
preprocessing, a preprocessing translation unit is called a {\it
  translation unit}. So a preprocessing translation unit can be
obntained by running \type{gcc -E filename.c} and such a translation
unit can be obtained by running \type{gcc -c filename.c}. When we run
\type{gcc -E filename.c} is sent to STDOUT i.e. monitor or screen so
if you want to see the output then you need to redirect it to a
file. You can give a command like \type{gcc -E
  filename.c>filename.extension} so that new file with name
\type{filename.extension} is saved to hard disk and can be viewed
later. When you give command \type{gcc -c filename.c} it automatically
generates corresponding \type{filename.o} unless you specify something
like \type{gcc -c filename.c -o anotherfilename.o}. This however is
not a good idea and you should refrain from it as it may lead to
filename clashes sooner or later if you try this manual option. So any
*.o file is a translation unit. It is also called object code.
